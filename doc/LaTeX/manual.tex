\documentclass[twoside,a4paper]{refart}
\usepackage{makeidx}
\usepackage{ifthen}

\def\bs{\char'134 } % backslash in \tt font.
\newcommand{\ie}{i.\,e.,}
\newcommand{\eg}{e.\,g..}
\DeclareRobustCommand\cs[1]{\texttt{\char`\\#1}}

\title{Mikko's Open Source Stacker for astronomical images}
\author{Mikko Laine}

\date{2014-09-08}
\emergencystretch1em  %

\pagestyle{myfootings}
\markboth{Mikko's Open Source Stacker for astronomical images}%
         {Mikko's Open Source Stacker for astronomical images}

\makeindex 

\setcounter{tocdepth}{2}

\begin{document}

\maketitle

\begin{abstract}
	\textit{Mosstack} is an open source program for calibrating,
	aligning and stacking astronomical photographs taken with a DSLR
	camera. This manual explains in detail all the technologies and
	algorithms used in Mosstack as well as all the commands and
	features of the program.
\end{abstract}


% Place holder for more text

\tableofcontents

\newpage


%%%%%%%%%%%%%%%%%%%%%%%%%%%%%%%%%%%%%%%%%%%%%%%%%%%%%%%%%%%%%%%%%%%%

\section{Introduction}

\subsection{Astrophotography}
\label{astrophotography}

\newpage

\section{Manual}
\label{manual}
\index{manual}

\subsection{General}
\index{general}

Mosstack started as a command line interface (CLI) program. Initially it was because I had no experience on programming
graphical user interfaces, but soon I realized the power of CLI. Now that I writing this (0.6~rc1 just came out) I have 
to say that doing a GUI right from the beginning would have made things a lot easier. GUI has it's own difficulties, but
CLI has to be built from scratch and many different user errors has to be dealt with.

Nevertheless benefits of CLI became apparent during testing and debugging. I can now write simple scripts about the
stacking process. Also automating the process is possible with CLI scripts. By controlling the camera with computer
scripts I can easily take calibration photos and process them ready with just a click. This kind of features are of course 
possible to make for GUI as well but it would reguire for me to anticipate what users want.

\subsection{Command line interface}
\label{cli}

Command line interface (CLI) is actually just command line arguments parsed with \texttt{mosstack} executable. The command
\texttt{mosstack} does nothing by itself except prints short help. Anything user wishes to do is done with command line
arguments. There's a list of arguments in section \ref{list of commands} and each individual command is explained in sections
following that.

All commands are done on the active project. Project can be initialized with command \texttt{init} (section \ref{init}) and
active project can be changed with the command \texttt{set} (section \ref{set}). There can be multiple projects in progress,
but only one instance of \texttt{mosstack} is supported at the time. CLI version of Mosstack does not multithread at the
moment (version 0.6), but the GUI version \texttt{mosstackgui} does.

The current project file is always printed when \texttt{mosstack} is run.

\subsubsection{CLI usage}
\label{cli usage}
\index{CLI usage}

All commands for \texttt{mosstack} follow the same pattern:

\begin{verbatim}
 mosstack <command> <arguments>
\end{verbatim}

There are couple of variations about \texttt{<arguments>}. All possible arguments are explained in sections after \ref{list of commands}.
Here are a couple examples:

\begin{verbatim}
 mosstack dir /path/to/photos/2014-10-22/Andromeda light
\end{verbatim}

Argument \texttt{dir} means adding a whole directory of files in to the active project. Mosstack tries to recognize every file
in the directory and adds each one it can decode with dcraw so be sure there are no extra files you don't wish to add.

Argument \texttt{dir} is followed by an Unix path and finally the type of frames. Supported keywords are light, bias, dark and
flat.

Another example:

\begin{verbatim}
 mosstack subtract light bias
\end{verbatim}

The argument \texttt{subtract} is used for calibrating the frames. In this example the stacked ''master'' bias frame is
subtracted from all the light frames. Before this command the master bias has to be ready and there has to be light frames
in the project.


\subsubsection{List of commands}
\label{list of commands}
\index{list of commands}

\begin{tabular}{|l|l|}
\hline
help        & print help \\ \hline
init        & initialize a new project \\ \hline
list        & list settings \\ \hline
set         & change settings \\ \hline
dir         & add whole directory of images\\ \hline
file        & add a single image\\ \hline
debayer     & debayer frames\\ \hline
register    & register frames\\ \hline
stack       & stack frames\\ \hline
subtract    & subtract image from a set\\ \hline
divide      & divide set by an image\\ \hline
bias        & subtract constant int from a set \\ \hline
master      & add master frame \\ \hline
size        & show projects size on disc \\ \hline
clean       & remove temporary files from project \\ \hline

\end{tabular}

\subsubsection{help}
\label{help}

Print the long help. Long help is mostly the section \ref{cli} from this manual. My plan is to make it generate
straight from this \LaTeX document.

How to run: 

\begin{verbatim}
 mosstack help
\end{verbatim}



\subsubsection{init}
\label{init}

Initialize the project. Mosstack always does everything for the active project and this command is used to create
one.

For example

\begin{verbatim}
 mosstack init Andromeda
\end{verbatim}

creates a new .project file in mosstack's working directory. This file holds information about progress of the
project as well as locations to all the files project uses.

Each diffrerent data set should be a separate project. One project can be used to stack the same data in different
ways, for example a maximum stack to find satellite trails and a sigma median stack to do the ''real'' final image.
A project leaves all the temporary files behind so changing settings and continuing on any point of the process should
be possible.

Since all the temporary data is left behind, in the end user should clean that with command \texttt{clean} \ref{clean}.

\subsubsection{list}
\label{list}
List settings. Running just

\begin{verbatim}
 mosstack list
\end{verbatim}

gives a list of possible settings. At the moment there are four settings:
\begin{itemize}
 \item debayer - How to debayer CFA images (see Sections \ref{debayering} and \ref{debayeringmath})
 \item matcher - How to find matching stars on different photos (see Sections \ref{registering} and \ref{registeringmath})
 \item transformer - How to perform affine transformations.
 \item stack - How to calculate image stacks (see Sections \ref{stacking} and \ref{stackingmath})
\end{itemize}

Running

\begin{verbatim}
 mosstack list <setting>
\end{verbatim}

for example

\begin{verbatim}
 mosstack list debayer
\end{verbatim}

gives a list of different debayering algorithms available. The current setting is also printed.

\subsubsection{set}
\label{set}

Change settings or the active project.

Command
\begin{verbatim}
 mosstack set project <project name>
\end{verbatim}

changes active project to given name. Project has to be initiated and the .project file has to be in working directory.
The option \texttt{<project name>} is given without path or suffix, exactly the same way it was written to command 
\texttt{init}. There's no other way to see available projects than to \texttt{ls *.project} in mosstack's working 
directory.

Settings are changed with options seen with command \texttt{list} as described in section \ref{list}. First see what
options there are available with

\begin{verbatim}
 mosstack list <setting>
\end{verbatim}

and change them with

\begin{verbatim}
 mosstack set <setting> <option>
\end{verbatim}

For example

\begin{verbatim}
 mosstack list debayer
\end{verbatim}

prints out

\begin{verbatim}
Options for the setting "debayer" are:

1.  BilinearOpenCl
2.  VNGOpenCl
3.  BilinearCython
4.  VNGCython

Current setting is "VNGCython".

\end{verbatim}

Now the setting can be change with either one of

\begin{verbatim}
 mosstack set debayer VNGOpenCl
 mosstack set debayer 2
\end{verbatim}

The text has to be written exactly as in the example, so number might be a better choice. Although for scripting the
text is better choice since it's not guaranteed that the options are always in the same order. They should be, but
there's nothing to check that.

\subsubsection{dir}
\subsubsection{file}
\subsubsection{debayer}
\label{debayering}
\subsubsection{register}
\label{registering}
\subsubsection{stack}
\label{stacking}
\subsubsection{subtract}
\subsubsection{divide}
\subsubsection{bias}
\subsubsection{master}
\subsubsection{size}
\subsubsection{clean}
\label{clean}




\subsection{Graphical user interface}

\newpage
\section{Mathematics}
\label{mathematics}

The reason for writing the manual on \LaTeX is this section. A big part of this project for me is to understand myself the
consepts and mechanisms of stacking. How it all works? Writing it out is one more challenge for me. \emph{If you can't 
explain it simply, you don't understand it well enough}. This quote has been credited to Einstein, but seems like there's
no source to back that up. Whoever said it is on to something.



\subsection{Debayering}
\label{debayeringmath}

\subsection{Registering}
\label{registeringmath}

\subsection{Stacking}
\label{stackingmath}

\printindex

\end{document}