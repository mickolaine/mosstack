\documentclass[twoside,a4paper]{refart}
\usepackage{makeidx}
\usepackage{ifthen}

\def\bs{\char'134 } % backslash in \tt font.
\newcommand{\ie}{i.\,e.,}
\newcommand{\eg}{e.\,g..}
\DeclareRobustCommand\cs[1]{\texttt{\char`\\#1}}

\title{Mikko's Open Source Stacker for astronomical images}
\author{Mikko Laine}

\date{2014-12-04}
\emergencystretch1em  %

\pagestyle{myfootings}
\markboth{Mikko's Open Source Stacker for astronomical images}%
         {Mikko's Open Source Stacker for astronomical images}

\makeindex 

\setcounter{tocdepth}{2}

\begin{document}

\maketitle

\begin{abstract}
	\textit{Mosstack} is an open source program for calibrating,
	aligning and stacking astronomical photographs taken with a DSLR
	camera. This manual explains in detail all the technologies and
	algorithms used in Mosstack as well as all the commands and
	features of the program.
	
	This manual is far from finished. I'm going to release it anyways
	because using the command line interface requires some assistance
	for most people (including me). 
\end{abstract}


% Place holder for more text

\tableofcontents

\newpage


%%%%%%%%%%%%%%%%%%%%%%%%%%%%%%%%%%%%%%%%%%%%%%%%%%%%%%%%%%%%%%%%%%%%

\section{Introduction}


\subsection{Astrophotography}
\label{astrophotography}
This section will cover the basics about astronomical photography


\newpage

\section{Manual}
\label{manual}
\index{manual}

\subsection{General}
\index{general}

Mosstack started as a command line interface (CLI) program. Initially it was because I had no experience on programming
graphical user interfaces, but soon I realized the power of CLI. Now that I writing this (0.6~rc1 just came out) I have 
to say that doing a GUI right from the beginning would have made things a lot easier. GUI has it's own difficulties, but
CLI has to be built from scratch and many different user errors has to be dealt with.

Nevertheless benefits of CLI became apparent during testing and debugging. I can now write simple scripts about the
stacking process. Also automating the process is possible with CLI scripts. By controlling the camera with computer
scripts I can easily take calibration photos and process them ready with just a click. This kind of features are of course 
possible to make for GUI as well but it would reguire for me to anticipate what users want.

\subsection{Command line interface}
\label{cli}

Command line interface (CLI) is actually just command line arguments parsed with \texttt{mosstack} executable. The command
\texttt{mosstack} does nothing by itself except prints short help. Anything user wishes to do is done with command line
arguments. There's a list of arguments in section \ref{list of commands} and each individual command is explained in sections
following that.

All commands are done on the active project. Project can be initialized with command \texttt{init} (section \ref{init}) and
active project can be changed with the command \texttt{set} (section \ref{set}). There can be multiple projects in progress,
but only one instance of \texttt{mosstack} is supported at the time. CLI version of Mosstack does not multithread at the
moment (version 0.6), but the GUI version \texttt{mosstackgui} does.

The current project file is always printed when \texttt{mosstack} is run.

\subsubsection{CLI usage}
\label{cli usage}
\index{CLI usage}

All commands for \texttt{mosstack} follow the same pattern:

\begin{verbatim}
 mosstack <command> <arguments>
\end{verbatim}

There are couple of variations about \texttt{<arguments>}. All possible arguments are explained in sections after \ref{list of commands}.
Here are a couple examples:
 
\begin{verbatim}
 mosstack dir /path/to/photos/2014-10-22/Andromeda light
\end{verbatim}

Argument \texttt{dir} means adding a whole directory of files in to the active project. Mosstack tries to recognize every file
in the directory and adds each one it can decode with dcraw so be sure there are no extra files you don't wish to add.

Argument \texttt{dir} is followed by an Unix path and finally the type of frames. Supported keywords are light, bias, dark and
flat.

Another example:

\begin{verbatim}
 mosstack subtract light bias
\end{verbatim}

The argument \texttt{subtract} is used for calibrating the frames. In this example the stacked ''master'' bias frame is
subtracted from all the light frames. Before this command the master bias has to be ready and there has to be light frames
in the project.


\subsubsection{List of commands}
\label{list of commands}
\index{list of commands}

\begin{tabular}{|l|l|}
\hline
help        & print help \\ \hline
init        & initialize a new project \\ \hline
list        & list settings \\ \hline
set         & change settings \\ \hline
dir         & add whole directory of images\\ \hline
file        & add a single image\\ \hline
frames      & list frames\\ \hline
remove      & remove frame \\ \hline 
reference   & change reference frame \\ \hline
debayer     & debayer frames\\ \hline
register    & register frames\\ \hline
crop        & crop frames\\ \hline
stack       & stack frames\\ \hline
subtract    & subtract image from a set\\ \hline
divide      & divide set by an image\\ \hline
biaslevel   & subtract constant int from a set \\ \hline
master      & add master frame \\ \hline
size        & show projects size on disc \\ \hline
clean       & remove temporary files from project \\ \hline

\end{tabular}

\subsubsection{help}
\label{help}
\index{help}

Print the long help. Long help is mostly the section \ref{cli} from this manual. My plan is to make it generate
straight from this \LaTeX document.

How to run: 

\begin{verbatim}
 mosstack help
\end{verbatim}



\subsubsection{init}
\label{init}
\index{init}

Initialize the project. Mosstack always does everything for the active project and this command is used to create
one.

For example

\begin{verbatim}
 mosstack init Andromeda
\end{verbatim}

creates a new .project file in mosstack's working directory. This file holds information about progress of the
project as well as locations to all the files project uses.

Each diffrerent data set should be a separate project. One project can be used to stack the same data in different
ways, for example a maximum stack to find satellite trails and a sigma median stack to do the ''real'' final image.
A project leaves all the temporary files behind so changing settings and continuing on any point of the process should
be possible.

Since all the temporary data is left behind, in the end user should clean that with command \texttt{clean} \ref{clean}.

\subsubsection{list}
\label{list}
\index{list}
List settings. Running just

\begin{verbatim}
 mosstack list
\end{verbatim}

gives a list of possible settings. At the moment there are four settings:
\begin{itemize}
 \item debayer - How to debayer CFA images (see Sections \ref{debayering} and \ref{debayeringmath})
 \item matcher - How to find matching stars on different photos (see Sections \ref{registering} and \ref{registeringmath})
 \item transformer - How to perform affine transformations.
 \item stack - How to calculate image stacks (see Sections \ref{stacking} and \ref{stackingmath})
\end{itemize}

Running

\begin{verbatim}
 mosstack list <setting>
\end{verbatim}

for example

\begin{verbatim}
 mosstack list debayer
\end{verbatim}

gives a list of different debayering algorithms available. The current setting is also printed.

\subsubsection{set}
\label{set}
\index{set}
\index{settings}

Change settings or the active project.

Command
\begin{verbatim}
 mosstack set project <project name>
\end{verbatim}

changes active project to given name. Project has to be initiated and the .project file has to be in working directory.
The option \texttt{<project name>} is given without path or suffix, exactly the same way it was written to command 
\texttt{init}. There's no other way to see available projects than to \texttt{ls *.project} in mosstack's working 
directory.

Settings are changed with options seen with command \texttt{list} as described in section \ref{list}. First see what
options there are available with

\begin{verbatim}
 mosstack list <setting>
\end{verbatim}

and change them with

\begin{verbatim}
 mosstack set <setting> <option>
\end{verbatim}

For example

\begin{verbatim}
 mosstack list debayer
\end{verbatim}

prints out

\begin{verbatim}
Options for the setting "debayer" are:

1.  BilinearOpenCl
2.  VNGOpenCl
3.  BilinearCython
4.  VNGCython

Current setting is "VNGCython".

\end{verbatim}

Now the setting can be change with either one of

\begin{verbatim}
 mosstack set debayer VNGOpenCl
 mosstack set debayer 2
\end{verbatim}

The text has to be written exactly as in the example, so number might be a better choice. Although for scripting the
text is better choice since it's not guaranteed that the options are always in the same order. They should be, but
there's nothing to check that.

\subsubsection{dir}
\index{dir}
Add whole directory of files to the project. Command works by listing contents of a Unix path, absolute or relative,
and checking each file with \texttt{dcraw -i}. If file is recognized as a DSLR Raw photo, it is added to project. Frame
type (light, bias, dark or flat) must also be defined.

Example

\begin{verbatim}
 mosstack dir /path/to/photos/2014-10-22/Andromeda light
\end{verbatim}

adds all files in path \textit{/path/to/photos/2014-10-22/Andromeda} to project as light frames. Path can also be
a relative path:

\begin{verbatim}
 mosstack dir 2014-10-22/Bias bias
\end{verbatim}

Be sure not to give wild cards * in your path since this is not ''add multiple files'' but ''add directory''.


\subsubsection{file}
\index{file}
Add a file to active project. Does not support wild cards (0.6, maybe later will) so each file must be added one at
a time. Like this:

\begin{verbatim}
 mosstack file /path/.../2014-10-22/Andro/IMG_5423.cr2 light
\end{verbatim}

As with \texttt{dir} command, the path can be either relative or absolute and command must end with frame type (light,
dark, bias, flat).

The file will be tested with \texttt{dcraw -i} and if DCRaw can decode it, it will be added to project.

\subsubsection{frames}
\index{frames}
List all the frames of given type. This is required for commands \texttt{remove} and \texttt{reference}.

Command is run with:
\begin{verbatim}
 mosstack frames <ftype>
\end{verbatim}

where \texttt{<ftype>} is frame type to list.

Example: List all the light frames:

\begin{verbatim}
 mosstack frames light
\end{verbatim}


\subsubsection{remove}
\index{remove}
Remove frame from the project. Use command \texttt{frames} to list frames and see their identifying numbers.

Command is run with:
\begin{verbatim}
 mosstack remove <ftype> <number>
\end{verbatim}

where \texttt{<ftype>} is frame type and \texttt{<number>} the identifying number for the frame.

Example: Check light frames and remove number 12.
\begin{verbatim}
 mosstack frames light
 mosstack remove light 12
\end{verbatim}


\subsubsection{reference}
\index{reference}
Change the reference frame. When adding frames the first one is always selected the reference frame.
This means all the other frames are matched to its stars and aligned as such.

Command is run with:
\begin{verbatim}
 mosstack reference <number>
\end{verbatim}

where \texttt{<number>} is identifying number for the frame. Check with command \texttt{frames}. The frame
type is not defined since setting reference frame is sensible only for light frames. This will operate
on light only.

\subsubsection{debayer}
\label{debayering}
\index{debayer}
Batch of frames will be debayered. Debayering process is explained in Section \ref{debayeringmath}.

Command is run with
\begin{verbatim}
 mosstack debayer <fphase>
\end{verbatim}

where \texttt{<phase>} is identifier for the batch.

Example: Run debayer for light frames

\begin{verbatim}
 mosstack debayer light
\end{verbatim}

Example: Run debayer for calibrated light frames saved with identifier \textit{calib}

\begin{verbatim}
 mosstack debayer calib
\end{verbatim}

Debayer will save batch with identifier \textit{rgb}.

There are two different language for debayering algorithms. These operations are quite processor intensive
so pure Python isn't a good choice. NumPy itself can't help either. There are two implementations: Cython and 
PyOpenCL.

Cython uses CPU to do the math. There's no multithreading and the process takes about 10 seconds for a 12\,Mpix
data on AMD FX-6300. This uses only one core so multithreading could be possible. It's just not implemented
yet on CLI.

PyOpenCL uses GPU for the math. This is \emph{fast}. It takes about 0.2 seconds to debayer a frame. Except 
that there is a 5 second overhead for manipulating the NumPy array to right alignment, transferring the 
data to GPU and back. It might be possible to manipulate the data on GPU and take some overhead off, but
this is the situation at version 0.6. Nevertheless it's faster than same with Cython.

Support for OpenCL doesn't work out of the box with Ubuntu. With Gentoo it works reasonably well
with programs from official distribution, but with Ubuntu 14.04 requires 3rd party packages. 

Multithreading does not work at all with PyOpenCL. Not even in GUI. Seems like it's just not possible. It
might be possible to multithread all the overhead stuff and just queue math itself, but there's no plans
for that yet.

\subsubsection{register}
\label{registering}
\index{register}
Batch of frames will be registered. This means aligning them for stacking. All of the current registering
methods work the same:
\begin{itemize}
 \item find stars
 \item match stars
 \item calculate affine transformation
 \item do the affine transformation
\end{itemize}

Further information about registering process is explained in Section \ref{registeringmath}.

Command is run with
\begin{verbatim}
 mosstack register <fphase>
\end{verbatim}

where \texttt{<fphase>} is identifier for the batch.

Example: Run register for calibrated and debayered light frames saved with identifier \textit{rgb}
\begin{verbatim}
 mosstack register rgb
\end{verbatim}

Register will save the batch with identifier \textit{reg}.

Currently there's no choice on the algorithms.

\subsubsection{crop}
\label{cropping}
\index{crop}
Batch of registered images will be cropped by given XY-range. Coordinates are given as pixels from upper
left corner. I recommend doing all the croppings with the graphical user interface.

Command is run with:
\begin{verbatim}
 mosstack crop <fphase> x0, x1, y0, y1
\end{verbatim}

where x0, x1, y0 and y1 are coordinates limiting a rectangular area and \texttt{<fphase>} identifier for 
frames to crop. Mostly this is \textit{reg} since only aligned images are good for cropping.

Example: Crop a rectangle limited by corners (300, 200) and (1800, 1300)

\begin{verbatim}
 mosstack crop reg 300 1800 200 1300
\end{verbatim}

Images are saved with identifier \textit{crop}

\subsubsection{stack}
\label{stacking}
\index{stack}
Batch of frames will be stacked with the selected stacking algorithm. Note that the batch should be aligned
before this.

Command is run with

\begin{verbatim}
 mosstack stack <fphase>
\end{verbatim}

where \texttt{<fphase>} is identifier for the batch. For light frames this usually is \textit{reg}, but for
calibration frames just the name \textit{dark}, \textit{flat} or \textit{bias}.

Example: Run stack for registered frames saved with identifier \textit{reg}
\begin{verbatim}
 mosstack stack reg
\end{verbatim}

The result image will be saved with identifier \textit{master}. Full name of the resulting files are printed
after successful stacking.

Note that stacking algorithm can be changed during project. If you want to stack calibration frames with
average value and light frames with eg. sigma median, just use \texttt{set} to change stacker before running
stack. See tutorial \ref{tutorial} for examples.

\subsubsection{subtract}
\index{subtract}
Subrtact frame from all the frames in batch. This is used with bias and dark frame calibration.

Command is run with:

\begin{verbatim}
 mosstack subtract <batch> <calib>
\end{verbatim}

where \texttt{<batch>} is identifier for batch to subtract from and \texttt{<calib>} identifier for master 
calibration frame. Note that the requested master must exist, or the command will fail.

Example: Subtract master bias from dark frames.

\begin{verbatim}
 mosstack subtract dark bias
\end{verbatim}

Resulting light images will be saved with identifier \textit{calib}. Calibrated dark and flat will be overwritten
with the same name.

Example: Stack bias frames, calibrate and stack dark frames and calibrate light frames.

\begin{verbatim}
 mosstack stack bias
 mosstack subtract dark bias
 mosstack stack dark
 mosstack subtract light bias
 mosstack subtract calib dark
\end{verbatim}

Note that first time you subtract from light frames the identifier is \textit{light} but after that it's \textit{calib}.
Operating on \textit{light} always takes the original files. Use this if you want to undo subtractions.



\subsubsection{divide}
\index{divide}
Divide frame from all the frames in batch. This is used with flat frame calibration.

Command is run with:

\begin{verbatim}
 mosstack divide <batch> <calib>
\end{verbatim}

where \texttt{<batch>} is identifier for batch to subtract from and \texttt{<calib>} identifier for master 
calibration frame. There really is no reason to do anything but dividing light (identifier \textit{light}
or \textit{calib}) with master flat frame.

Example: Divide calibrated light frames with master flat frame.

\begin{verbatim}
 mosstack divide calib flat
\end{verbatim}

Result will be saved with identifier \textit{calib}. 


\subsubsection{biaslevel}
\index{biaslevel}
Set \textit{bias level} for batch. This works like bias frame calibration, but subtracts a constant value
from all the pixels.

Command is run with:
\begin{verbatim}
 mosstack bias <batch> <value>
\end{verbatim}

where \texttt{<batch>} is identifier for the batch and \texttt{<value>} is the value to subtract from pixels.

Example: Subtract bias level 21 from light frames.

\begin{verbatim}
 mosstack bias light 21
\end{verbatim}

The resulting batch will be saved with identifier \textit{calib}, just like if they were calibrated with bias
frame.


\subsubsection{master}
\index{master}
Add a pre-existing master frame to the project. With this you don't need to remake the calibration frames for
all projects from same photo session.

Command is run with:
\begin{verbatim}
 mosstack master <path> <type>
\end{verbatim}

where \texttt{<path>} is a Unix path to file and \texttt{<type>} is type of the calibration frame (\textit{dark},
\textit{flat} or \textit{bias}).

Example: Add master flat frame to the project.

\begin{verbatim}
 mosstack master /path/to/saved/flat_2014-10-20.fits flat
\end{verbatim}

Formats Fits and Tiff are supported.


\subsubsection{size}
\index{size}
Tell the size of all the files in project. Mosstack creates a lot of temporary files required only for the next
step in processing. Depending on the number and size of source files, a project can easily take several gigabytes
of space.

Command is run with:
\begin{verbatim}
 mosstack size
\end{verbatim}

That's it. It prints out the size in the most convinient unit. Most likely that will be GiB, but for small projects
maybe MiB.

\subsubsection{clean}
\label{clean}
\index{clean}
Remove all the temporary files. This is a useful command to run after the project is done. \texttt{Clean} removes
all the temprary files leaving only ones labeled \textit{master} and the project file. Everything can be easily
run again without adding all the files.

Command is run with:
\begin{verbatim}
 mosstack clean
\end{verbatim}



\subsection{Tutorial}
\label{tutorial}
\index{tutorial}
The best thing about command line interface is the possibilities of scripting and automating stuff. It's possible
here as well. This section presents some copy \& paste examples on how to use the command line interface of Mosstack.

\subsubsection{Simple project}

The first case is a simple project with default settings and no calibration frames. Replace \texttt{<lightpath>}
with a Unix path to the light frames wherever it's used.

The workflow is following:
\begin{itemize}
 \item Create project. Always do this first since otherwise it tries to use previous project and most likely will
       fail
 \item Add files. Use commands \texttt{dir} and \texttt{file}, whatever suits you. At this time wildcards are not
       supported so adding several files might be a pain. I recommend adding whole directory, but in that case
       the files must be the only frames in the directory. When scripting, a Bash \textit{for i in ...} loop
       might be handy.
 \item Calibrating. Nothing now but this is the right spot to do that.
 \item Debayer. This decodes the color filter array (CFA) images into three colour (RGB) images. Do this for 
       color output. Actually do this always if you don't know what it is.
 \item Register. Match stars and align images accordingly. This is required for a succesful stack. If your frames
       are already aligned this is only a waste of time.
 \item Stack. When all the preparations are done, do the stack.
 \item Postprocessing. Mosstack does not balance colours, do sharpening or adjust tone curves. For all that
       you need a separate program. I recommend Darktable.
\end{itemize}

Translated to Mosstack, that is:

\begin{verbatim}
 mosstack project Andromeda
 mosstack dir <lightpath>
 mosstack debayer light
 mosstack register rgb
 mosstack stack reg
\end{verbatim}

\subsubsection{With calibration}

Let's use some bias and flat frames. Use averaging to stack them and sigma median for the light stack. That means
the settings must be changed during process.

Calibration is explained theoretically in section \ref{calibrationmath}.

Stuff to do for calibration:
\begin{itemize}
 \item Change stacker to Mean 
 \item Stack bias
 \item Subtract bias from flat
 \item Stack flat
 \item Chance stacker to desired option for light frames
 \item Calibrate light frames
\end{itemize}

Otherwise this goes along the Simple project

\begin{verbatim}
 mosstack project Andromeda
 mosstack dir <lightpath>
 mosstack dir <biaspath>
 mosstack dir <flatpath>
 mosstack set stack Mean
 mosstack stack bias
 mosstack subtract flat bias
 mosstack stack flat
 mosstack set stack SigmaMedian
 mosstack subtract light bias
 mosstack divide calib flat
 mosstack debayer light
 mosstack register rgb
 mosstack stack reg
\end{verbatim}

As you see, there are many lines more compared to Simple project without calibration.

\subsubsection{More?}

Under construction

\subsection{Graphical user interface}

Under construction

\newpage
\section{Mathematics}
\label{mathematics}

The reason for writing the manual on \LaTeX is this section. A big part of this project for me is to understand myself the
consepts and mechanisms of stacking. How it all works? Writing it out is one more challenge for me. \emph{If you can't 
explain it simply, you don't understand it well enough}. This quote has been credited to Einstein, but seems like there's
no source to back that up. Whoever said it is on to something.

\subsection{Calibrating}
\label{calibrationmath}

\subsection{Debayering}
\label{debayeringmath}

\subsection{Registering}
\label{registeringmath}

\subsection{Stacking}
\label{stackingmath}

\printindex

\end{document}